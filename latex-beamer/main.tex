\documentclass[green,10pt]{beamer}
\usepackage{pgfshade}
\usetheme{Warsaw}
\setbeamercovered{transparent}
\usecolortheme{seahorse}

\beamertemplatenavigationsymbolsempty

\usepackage{ngerman}
\usepackage[utf8]{inputenc} % Linux-Systeme

\newcommand{\vtiny}{\fontsize{6}{1}\selectfont}
\usepackage[absolute,overlay]{textpos}

\newcommand{\zitat}[2]{
  \textit{\frqq#1 \flqq}\\
  \begin{flushright}
    -- #2
  \end{flushright}
}
\newcommand{\quelle}[1]{
  \begin{textblock*}{\linewidth}(1cm,.92\paperheight)
    \vtiny Quelle: #1
  \end{textblock*}
}
\newcommand{\symbole}[1]{
  \begin{textblock*}{\linewidth}(1cm,.79\paperheight)
    \parbox{0.75\linewidth}{\vtiny #1}
  \end{textblock*}
}
\newcommand{\name}[1]{
  \textsc{#1}
}
\newcommand{\adtef}{A Dynamical Theory of the Electromagnetic Field}

\title[Maxwell: A Dynamical Theory of the Electromagnetic Field]{J.\,C.\,Maxwell: A Dynamical Theory of the Electromagnetic Field}
\subtitle{Wegweisende Originalarbeiten in der Geschichte der Physik}

\author[Sebastian Fiedlschuster \texttt{<sebastian@fiedlschuster.de>}]{Sebastian Fiedlschuster \\ \tiny\texttt{sebastian@fiedlschuster.de}}
\institute{Friedrich-Alexander-Universit"at Erlangen-N"urnberg \\ Department f"ur Physik }
\logo{\includegraphics[height=1.5cm]{img/fau.jpg}}

\date{11.\,Februar 2009}

\begin{document}

  \frame{\titlepage}
  \frame{
    \frametitle{Vortragsmaterialien}
    \begin{center}
      Vortragsmaterialien können von folgender Adresse\\ heruntergeladen werden:\\ \vspace{0.3cm}
      \url{www.fiedlschuster.de/physik/}
    \end{center}
    
  }

\section[Einleitung]{Einleitung}
  \frame{
    \frametitle{Einleitung}
    \zitat{So soll ich denn mit saurem Schweiß\\Euch lehren, was ich selbst nicht weiß.\\
      Dieses Wort Fausts ist wohl niemandem so sehr aus der Seele gesprochen wie dem, der über die wahre Natur der Elektrizität vortragen will.}{\name{L.\,Boltzmann} in seinem Buch über die Maxwellsche\\ Theorie des Elektromagnetismus.}
    \quelle{Simonyi, S.\,349}
  }
  \frame{
    \frametitle{Inhaltsverzeichnis}
    \tableofcontents[hideallsubsections,sections=2-6]
  }

\section{Biographie Maxwells}
  \frame{
    \frametitle{Inhaltsverzeichnis}
    \tableofcontents[sections=2-6,currentsection,hideallsubsections]
  }  
  \subsection{Frühe Jahre}
    \frame{
      \frametitle[J.\,C.\,Maxwell]{James Clerk Maxwell (1831--1879)}
      \begin{columns}[T]
        \begin{column}{0.3\textwidth}
          \begin{figure}
            \includegraphics[width=\textwidth]{img/maxwell3}
            \caption{Maxwell 1855 in Cambridge}
          \end{figure}
        \end{column}
        \begin{column}{0.6\textwidth}
          \begin{itemize}
            \item<1-> Geboren am 13.\,Juni 1831 in Edinburgh, Schottland
            \item<2-> 1844 \ \ Erste mathematische Abhandlung
            \item<3-> 1847--1850 \ \ Studium an der Universität Edinburgh
            \item<4-> 1849 \ \ Veröffentlichung für die "`Transactions of the Royal Society of Edinburgh"'
            \item<5-> 1850--1854 \ \ Studium an der Universität Cambridge
          \end{itemize}
        \end{column}
      \end{columns}
      \quelle{\url{http://upload.wikimedia.org/wikipedia/commons/a/ac/YoungJamesClerkMaxwell.jpg},\\ Stand: 10.\,Feb.\,2009}
    }
  \subsection{Mittlere Jahre}
    \frame{
      \frametitle[J.\,C.\,Maxwell]{James Clerk Maxwell (1831--1879)}
      \begin{columns}[T]
        \begin{column}{0.3\textwidth}
          \begin{figure}
            \includegraphics[width=\textwidth]{img/maxwell3}
            \caption{Maxwell 1855 in Cambridge}
          \end{figure}
        \end{column}
        \begin{column}{0.6\textwidth}
          \begin{itemize}
            \item<1-|alert@1-> 1856: \ \ \textit{On Faraday's Lines of Force}
            \item<2-> 1855--1872: \ \ Erforschung des Farbsehens
            \item<3-> 1856--1860: \ \ Lehrstuhl für Naturphilosophie, Aberdeen
            \item<4-> 1858: \ \ Hochzeit mit Katherine Mary Dewar
            \item<5-> 1859: \ \ Adams-Preis der Universität Cambridge: \textit{On the Stability of Saturn’s Rings}
          \end{itemize}
        \end{column}
      \end{columns}
     \quelle{\url{http://upload.wikimedia.org/wikipedia/commons/a/ac/YoungJamesClerkMaxwell.jpg},\\ Stand: 10.\,Feb.\,2009}     
    }
    \frame{
      \frametitle[J.\,C.\,Maxwell]{James Clerk Maxwell (1831--1879)}
      \begin{columns}[T]
        \begin{column}{0.3\textwidth}
          \begin{figure}
            \includegraphics[width=\textwidth]{img/maxwell2}
            \caption{J.\,C.\,Maxwell in den 1870ern}
          \end{figure}
        \end{column}
        \begin{column}{0.6\textwidth}
          \begin{itemize}
            \item<1-> 1860--1868: \ \ Lehrstuhl für Physik und Astronomie am Kings College, London
            \item<2-|alert@2-> 1861/62: \ \ \textit{On Physical Lines of Force}
            \item<3-|alert@3-> 1864: \ \ \textit{A Dynamical Theory of the Electromagnetic Field}
            \item<4-> 1866: \ \ Maxwell-Boltzmann-Gastheorie
            \end{itemize}
          \end{column}
      \end{columns}
      \quelle{Cavendish Laboratorium, University of Cambridge,\\
        \url{http://outreach.jach.hawaii.edu/pressroom/2006_jcmbday/Maxwell.jpg}, Stand: 10.\,Feb.\,2009}
    }
  \subsection{Späte Jahre}
    \frame{
      \frametitle[J.\,C.\,Maxwell]{James Clerk Maxwell (1831--1879)}
      \begin{columns}[T]
        \begin{column}{0.3\textwidth}
          \begin{figure}
            \includegraphics[width=0.8\textwidth]{img/maxwell4}
            \caption{James und Katherine Maxwell, 1869}
          \end{figure}
        \end{column}
        \begin{column}{0.6\textwidth}
          \begin{itemize}
            \item<1-> 1871: \ \ \textit{The Theory of Heat}
            \item<2-> 1871: \ \ Aufbau des Cavendish-Laboritoriums, Cambridge
            \item<3-|alert@3-> 1873: \ \ \textit{A Treatise on Electricity and Magnetism}
            %\item<4-> 1876: \ \ Einführung über Körper und deren Bewegung
            \item<4-> Tod: 5.\,November 1879, Cambridge
          \end{itemize}
        \end{column}
      \end{columns}
      \quelle{\url{http://upload.wikimedia.org/wikipedia/commons/3/33/JamesClerkMaxwell-KatherineMaxwell-1869.jpg}, Stand: 10.\,Feb.\,2009}
    }  
  \subsection{Zitate zu Maxwells Persönlichkeit}
    \frame{
      \frametitle[J.\,C.\,Maxwell]{James Clerk Maxwell (1831--1879)}
      \zitat{Maxwell liebte das Landleben und durchstreifte zu Pferde, begleitet von seinem Hund, gern die schottischen Moore. Die Jagd lehnte er aus Tierliebe ab. \\
      Mit den Leistungen anderer Wissenschaftler, auch wenn deren Ansichten ihm nicht ganz richtig vorkamen, ging er respektvoll um. }{\name{Karl von Meyenn}}
      \quelle{Meyenn, S.\,22}
    }
    \frame{
      \frametitle[J.\,C.\,Maxwell]{James Clerk Maxwell (1831--1879)}
      \zitat{Faraday und Maxwell gehörten -- auch hinsichtlich ihrer menschlichen Qualitäten -- zu den sympathischsten Persönlichkeiten der Physikgeschichte. In seiner Bescheidenheit ist Maxwell kaum von einem der großen Physiker übertroffen worden. Er hat seine eigenen Ergebnisse so dargestellt, als ob sie sich in einer mathematischen Formulierung der Gedanken Faradays erschöpften. Faraday seinerseits hat sich über die Fähigkeiten Maxwells sehr anerkennend geäußert.
      }{\name{Károly Simonyi}}
      \quelle{Simonyi, S. 345}
    }    
    
\section{A Dynamical Theory of the Electromagnetic Field}
  \frame{
    \frametitle{Inhaltsverzeichnis}
    \tableofcontents[sections=2-6,currentsection,hideallsubsections]
  }  
  \frame{
    \frametitle{Inhaltsverzeichnis}
    \tableofcontents[sections=3]
  }  
  \subsection{Ausgangspunkt Maxwells}
    \begin{frame}
      \frametitle{Ausgangpunkt Maxwells}
        \begin{itemize}[<+->]
          \item Kenntnis, Phantasie, mathematisches Werkzeug
          \item Bekannt: Induktion, Dielektrikum und Nahwirkung
          \item Øersted (1819): Einfluss bewegter Ladung auf Magneten
          \item Ampère (1820): Einfluss stromdurchflossener Leiter aufeinander
          \item Faraday: Nahwirkung statt Fernwirkung, Kraftlinien
          \item Fresnel (1822): Äther
        \end{itemize}

    \end{frame}

  \subsection{Einordnung des Artikels}
    \begin{frame}
      \frametitle{Einordnung des Artikels}
      \begin{itemize}
        \item<1-> 1856: \ \ \textit{On Faraday's Lines of Force}
        \item<2-> 1861/62: \ \ \textit{On Physical Lines of Force}
        \item<3-|alert@5-> 1864: \ \ \textit{A Dynamical Theory of the Electromagnetic Field}
        \item<4-> 1873: \ \ \textit{A Treatise on Electricity and Magnetism}
      \end{itemize}
    \end{frame}

  \subsection{Inhalt des Artikels}
    \frame{
      \frametitle{Inhalt des Artikels}
      \begin{itemize}
        \item<1-> Part I: Introductory
        \item<2-> Part II: On Electromagnetic Induction
        \item<3-| alert@8> Part III: General Equations of the Electromagnetic Field
        \item<4-> Part IV: Mechanical Actions in the Field
        \item<5-> Part V: Theory of Condensers
        \item<6-| alert@8> Part VI: Electromagnetic Theory of Light
        \item<7-> Part VII: Calculation of the Coefficients of Electromagnetic Induction
      \end{itemize}
    }
    \frame{
      \frametitle{I: Introductory}
      \begin{itemize}
        \item<1-|alert@1-> Part I: Introductory
        \item<1-> Part II: On Electromagnetic Induction
        \item<1-> Part III: General Equations of the Electromagnetic Field
        \item<1-> Part IV: Mechanical Actions in the Field
        \item<1-> Part V: Theory of Condensers
        \item<1-> Part VI: Electromagnetic Theory of Light
        \item<1-> Part VII: Calculation of the Coefficients of Electromagnetic Induction
      \end{itemize}
    }
    \frame{
      \frametitle{I: Introductory}
      \begin{itemize}[<+->]
        \item Überblick und Ergebnisse
        \item Ziel: Mechanismus zur Kraftübertragung anstelle einer Fernwirkung
        \item Titel des Aufsatzes
        \item Feldbegriff
        \item Energiebetrachtung
        \item Erkenntnisse aus Experimenten
        \item General Equations of the Electromagnetic Field
        \item Rolle des Lichtes
      \end{itemize}
    }
    \frame{
      \frametitle{I: Introductory}
      \zitat{
        This velocity is so nearly that of light, that it seems w have strong reason to conclude that light itself (including radiant heat, and other radiations if any) is an electromagnetic disturbance in the form of waves propagated through the electromagnetic field according to the electromagnetic laws.
      }{\name{Maxwell}}
      \quelle{Maxwell: \adtef, § 20}
    }
    
    \frame{
      \frametitle{II: On Electromagnetic Induction}
      \begin{itemize}
        \item<1-> Part I: Introductory
        \item<1-|alert@1-> Part II: On Electromagnetic Induction
        \item<1-> Part III: General Equations of the Electromagnetic Field
        \item<1-> Part IV: Mechanical Actions in the Field
        \item<1-> Part V: Theory of Condensers
        \item<1-> Part VI: Electromagnetic Theory of Light
        \item<1-> Part VII: Calculation of the Coefficients of Electromagnetic Induction
      \end{itemize}
    }  
    \frame{
      \frametitle{II: On Electromagnetic Induction}
      \begin{itemize}[<+->]
        \item Untersuchung von Induktionsphänomenen
        \item Systematisierung der Vorstellung von Feldlinien
      \end{itemize}
    }    
    
    \frame{
      \frametitle{III: General Equations of the Electromagnetic Field}
      \begin{itemize}
        \item<1-> Part I: Introductory
        \item<1-> Part II: On Electromagnetic Induction
        \item<1-|alert@1-> Part III: General Equations of the Electromagnetic Field
        \item<1-> Part IV: Mechanical Actions in the Field
        \item<1-> Part V: Theory of Condensers
        \item<1-> Part VI: Electromagnetic Theory of Light
        \item<1-> Part VII: Calculation of the Coefficients of Electromagnetic Induction
      \end{itemize}
    }  
    \frame{
      \frametitle{III: General Equations of the Electromagnetic Field}
      \begin{itemize}[<+->]
        \item Vorstellung der Größen und Gleichungen
        \item \zitat{I wish mereley to direct the mind of the reader to mechanical phenomena which will assist him in uderstanding the electrical ones. All such phrases in the present paper are to be considered as illustrative, not as explanatory.}{\name{Maxwell, § 73}}
          \quelle{Maxwell: \adtef}
        \item \zitat{The energy in electromagnetic phenomena is mechanical energy. The only question is, Where does it reside?}{\name{Maxwell}, § 74}
      \end{itemize}
    }    
    
    \frame{
      \frametitle{IV: Mechanical Actions in the Field}
      \begin{itemize}
        \item<1-> Part I: Introductory
        \item<1-> Part II: On Electromagnetic Induction
        \item<1-> Part III: General Equations of the Electromagnetic Field
        \item<1-|alert@1-> Part IV: Mechanical Actions in the Field
        \item<1-> Part V: Theory of Condensers
        \item<1-> Part VI: Electromagnetic Theory of Light
        \item<1-> Part VII: Calculation of the Coefficients of Electromagnetic Induction
      \end{itemize}
    }  
    \frame{
      \frametitle{IV: Mechanical Actions in the Field}
      \begin{itemize}[<+->]
        \item Mechanische Kräfte elektromagnetischer Phänomene
        \item Geschwindigkeit $v = 310\,740\,000\,$m/s
        \item Notiz zur Gravitation
      \end{itemize}
    }    
    
    \frame{
      \frametitle{V: Theory of Condensers}
      \begin{itemize}
        \item<1-> Part I: Introductory
        \item<1-> Part II: On Electromagnetic Induction
        \item<1-> Part III: General Equations of the Electromagnetic Field
        \item<1-> Part IV: Mechanical Actions in the Field
        \item<1-|alert@1-> Part V: Theory of Condensers
        \item<1-> Part VI: Electromagnetic Theory of Light
        \item<1-> Part VII: Calculation of the Coefficients of Electromagnetic Induction
      \end{itemize}
    }  
    \frame{
      \frametitle{V: Theory of Condensers}
      \begin{itemize}[<+->]
        \item Kapazität
        \item Elektrische Absorption
      \end{itemize}
    }        
    
    \frame{
      \frametitle{VI: Electromagnetic Theory of Light}
      \begin{itemize}
        \item<1-> Part I: Introductory
        \item<1-> Part II: On Electromagnetic Induction
        \item<1-> Part III: General Equations of the Electromagnetic Field
        \item<1-> Part IV: Mechanical Actions in the Field
        \item<1-> Part V: Theory of Condensers
        \item<1-|alert@1-> Part VI: Electromagnetic Theory of Light
        \item<1-> Part VII: Calculation of the Coefficients of Electromagnetic Induction
      \end{itemize}
    }  
    \frame{
      \frametitle{VI: Electromagnetic Theory of Light}
      \begin{itemize}[<+->]
        \item Welleneigenschaften elektromagnetischer Wellen
        \item Herleitung einer Wellengleichung
        \item Deckung mit der Lichtgeschwindigkeit
      \end{itemize}
    }    
    
    \frame{
      \frametitle{VII: Calculation of the Coefficients of Elm. Induction}
      \begin{itemize}
        \item<1-> Part I: Introductory
        \item<1-> Part II: On Electromagnetic Induction
        \item<1-> Part III: General Equations of the Electromagnetic Field
        \item<1-> Part IV: Mechanical Actions in the Field
        \item<1-> Part V: Theory of Condensers
        \item<1-> Part VI: Electromagnetic Theory of Light
        \item<1-|alert@1-> Part VII: Calculation of the Coefficients of Electromagnetic Induction
      \end{itemize}
    }  
    \frame{
      \frametitle{VII: Calculation of the Coefficients of Elm. Induction}
      \begin{itemize}[<+->]
        \item Magnetischer Fluss = Strom $\cdot$ Induktionskoeffizient
        \item Bestimmung von Induktionskoeffizienten
      \end{itemize}
    }    
    \frame{
      \frametitle{VII: Calculation of the Coefficients of Elm. Induction}
      \begin{itemize}
        \item<1-> Part I: Introductory
        \item<1-> Part II: On Electromagnetic Induction
        \item<1-> Part III: General Equations of the Electromagnetic Field
        \item<1-> Part IV: Mechanical Actions in the Field
        \item<1-> Part V: Theory of Condensers
        \item<1-> Part VI: Electromagnetic Theory of Light
        \item<1-|alert@1-> Part VII: Calculation of the Coefficients of Electromagnetic Induction
      \end{itemize}
    }  
    
\section{Maxwellgleichungen}
  \frame{
    \frametitle{Maxwells allgemeine Gleichungen}
    \begin{itemize}
      \item<1-> Part I: Introductory
      \item<1-> Part II: On Electromagnetic Induction
      \item<1-|alert@1-> Part III: General Equations of the Electromagnetic Field
      \item<1-> Part IV: Mechanical Actions in the Field
      \item<1-> Part V: Theory of Condensers
      \item<1-> Part VI: Electromagnetic Theory of Light
      \item<1-> Part VII: Calculation of the Coefficients of Electromagnetic Induction
    \end{itemize}
  } 
  \frame{
    \frametitle{Maxwells allgemeine Gleichungen}
    \tableofcontents[sections=2-6,currentsection,hideallsubsections]
  }  
  \subsection{Maxwells allgemeine Gleichungen}
    \newcommand{\magsymbole}{\symbole{Raumrichtungen $x,y,z$; elektrische Ströme $p,q,r$; elektrische Verschiebungen $f,g,h$;
        Totale Ströme $p',q',r'$; elektrische Kräfte $P,Q,R$; elm. Impulse $F,G,H$; magnetische Kräfte $\alpha,\beta,\gamma$;
        Koeffizient $\mu$ für magnetische Induktion; magnetisches Potential $\phi$; elektrisches Potential $\psi$; 
        elektrischer Widerstand $\rho_R$; elektrische Größe $e$ (Ladung); \\
        Stromdichte $j$; elektrische Flussdichte $D$ (=dielektrische Verschiebung); elektrische Feldstärke $E$;
        magnetische Flussdichte $B$; magnetische Feldstärke $H$; magnetisches Vektorpotential $A$; Ladungsdichte $\rho$;
    }} 
    \frame{
      \frametitle{Maxwells allgemeine Gleichungen}
      \magsymbole
      \[
        p' = p + \frac{df}{dt}
        , \hspace{2cm}
        j_\text{total} = j_\text{Leitung} + \frac{\partial D}{\partial t}
        \tag{A}
      \]
      \pause
      \[
        \mu\alpha = \frac{dH}{dy} - \frac{dG}{dz}
        , \hspace{2cm}
        B = \nabla \times A
        \tag{B}
      \]
      \pause
      \[
        \frac{d\gamma}{dy} - \frac{d\beta}{dz} = 4\pi p'
        , \hspace{2cm}
        \nabla \times H = j_\text{total}
        \tag{C}
      \]
      \pause
      \[
        P = \mu \left( \gamma\frac{dy}{dt} - \beta\frac{dz}{dt} \right) -\frac{dF}{dt} - \frac{d\psi}{dx}
        , \hspace{1 cm}
        E = \mu v \times H - \frac{\partial A}{\partial t} - \nabla \phi
        \tag{D}
      \]
    }      
    \frame{
      \frametitle{Maxwells allgemeine Gleichungen}
      \magsymbole
      \[
        P = k\,f 
        , \hspace{2cm}
        E = \frac{1}{\epsilon} D
        \tag{E}
      \]
      \pause
      \[
        P = - \rho_R \, p
        , \hspace{2cm}
        R = \frac{U}{I}
        \tag{F}
      \]
      \pause
      \[
        e + \frac{df}{dx} + \frac{dg}{dy} + \frac{dh}{dz} = 0
        , \hspace{1cm}
        \nabla D = \rho 
        \tag{G}
      \]
      \pause
      \[
        \frac{de}{dt} + \frac{dp}{dx} + \frac{dq}{dy} + \frac{dr}{dz} = 0
        , \hspace{1cm}
        \frac{\partial \rho}{\partial t} + \nabla j = 0
        % Was für ein j?
        % rho_l tauschen
        \tag{H}
      \]
    }  
    \frame{
      \frametitle{Maxwells allgemeine Gleichungen}
      Es handelt sich um 20 Gleichungen. \vspace{0.3cm}
      
      \begin{tabular}{lll}
        Three equations of  & Magnetic Force      & (B) \\
                            & Electric Currents   & (C) \\
                            & Electromotive Force & (D) \\
                            & Electric Elasticity & (E) \\
                            & Electric Resistance & (F) \\
                            & Total Currents      & (A) \\
        One equation of     & Free Electricity    & (G) \\
                            & Continuity          & (H)
      \end{tabular}    
    }
    \frame{
      \frametitle{Maxwells allgemeine Gleichungen}
      Es handelt sich um 20 Gleichungen.\\
      Die Gleichungen beinhalten 20 Variablen. \vspace{0.3cm}
      
      \begin{tabular}{ll}
        $F,G,H$                 & Electromagnetic Momentum                            \\
        $\alpha,\beta,\gamma$   & Magnetic Intensity                                  \\
        $P,Q,R$                 & Electromotive Force                                 \\
        $p,q,r$                 & Current due to true conduction                      \\
        $f,g,h$                 & Electric Displacement                               \\
        $p',q',r'$              & Total Current (including variation of displacement) \\
        $e$                     & Quantity of free Electricity                        \\
        $\psi$                  & Electric Potential                                    
      \end{tabular}
    } 
    \frame{
      \frametitle{Maxwells allgemeine Gleichungen}
      \zitat{These equations are therefore sufficient to determine all the quantities which occur in them, provided we know the conditions of the problem. In many questions, however, only a few of the equations are required.}{\name{Maxwell}}
      \quelle{Maxwell: \adtef, § 70}
    }     
    \frame{
      \frametitle{Maxwells allgemeine Gleichungen}
      \begin{tabular}{lll}
        Three equations of  & Magnetic Force      & (B) \\
                            & Electric Currents   & (C) \\
                            & Electromotive Force & (D) \\
                            & Electric Elasticity & (E) \\
                            & Electric Resistance & (F) \\
                            & Total Currents      & (A) \\
        One equation of     & Free Electricity    & (G) \\
                            & Continuity          & (H)
      \end{tabular}    
    }
    \frame{
      \frametitle{Maxwells allgemeine Gleichungen}
      \zitat{Damit wären die hauptsächlichen Beziehungen zwischen den betrachteten Größen gegeben. Einige dieser Größen können durch Kombination der Gleichungen eliminiert werden, aber es ist augenblicklich nicht unser Zweck, eine möglichst knappe mathematische Formulierung zu geben, sondern jede Beziehung, die wir kennen, zum Ausdruck zu bringen. In diesem Stadium der Untersuchung eine Größe, die irgendeine nützliche Idee ausdrückt, zu eliminieren, wäre eher ein Verlust als ein Gewinn.}{\name{Maxwell} im Treatise von 1873}
      \quelle{Frisius, S.\,200}
    }         
      
  \subsection{Herleitung der heutigen Maxwellgleichungen}
    \frame{
      \frametitle{Herleitung der heutigen Maxwellgleichungen}
      \framesubtitle{Gleichung 1}
      \magsymbole
      \[
        \nabla D = \rho 
        \tag{G}
      \]
      \begin{block}{}
        \[
          \Rightarrow \ \ \ \ \
          \nabla \ D = \rho \tag{M1: Coulomb}
        \]
      \end{block}
    }        
    \frame{
      \frametitle{Herleitung der heutigen Maxwellgleichungen}
      \framesubtitle{Gleichung 2}
      \magsymbole
      \[
        B = \nabla \times A
        \tag{B}
      \]
      \[
        \Rightarrow \ \ \
        \nabla \ B = \nabla \ (\nabla \times A) = 0
      \]
      \begin{block}{}
        \[
          \Rightarrow \ \ \ \ \
          \nabla \times B = 0 \tag{M2: Gauß}
        \]
      \end{block}
    }
    \frame{
      \frametitle{Herleitung der heutigen Maxwellgleichungen}
      \framesubtitle{Gleichung 3}
      \magsymbole
      \[
        E = \mu v \times H - \frac{\partial A}{\partial t} - \nabla \phi
        \tag{D}
      \]
      \[
        B = \nabla \times A
        \tag{B}
      \]
      \begin{block}{}
        \[
          \Rightarrow \ \ \ \ \
          \nabla \times E = - \frac{\partial}{\partial t} B\tag{M3: Faraday}
        \]
      \end{block}
    }
    \frame{
      \frametitle{Herleitung der heutigen Maxwellgleichungen}
      \framesubtitle{Gleichung 4}
      \magsymbole
      \[
        j_\text{total} = j + \frac{\partial D}{\partial t}
        \tag{A}
      \]
      \[
        \nabla \times H = j_\text{total}
        \tag{C}
      \]
      \begin{block}{}
        \[
          \Rightarrow \ \ \ \ \
          \nabla \times H = j + \frac{\partial}{\partial t} D\tag{M4: Ampère}
        \]
      \end{block}
    }
  
  \newcommand{\neuesymbole}{\symbole{\vspace{0.5cm} 
    $D$: Elektrische Flussdichte, dielektrische Verschiebung; $B$: Magnetische Flussdichte;
        \\ $E$: Elektrisches Feldstärke; $H$: Magnetische Feldstärke; $t$: Zeit;
        \\ $\rho$: Ladungsdichte; $j$: Stromdichte;
        \\ $\mu:=\mu_0\mu_r$: Permeabilität, mgn. Leitfähigkeit; 
        \\ $\epsilon:=\epsilon_0\epsilon_r$: Permittivität, dielektrische Leitfähigkeit
  }}
  
  \subsection{Maxwellgleichungen in heutiger Notation}
    \frame{
      \frametitle{Maxwellgleichungen in heutiger Notation}
      \begin{block}{Maxwellgleichungen}
        \begin{align}
          \nabla \ D &= \rho \tag{M1: Coulomb} \\
          \nabla \ B &= 0 \tag{M2: Gauß} \\
          \nabla \times E &= - \frac{\partial}{\partial t} B\tag{M3: Faraday} \\
          \nabla \times H &= j + \frac{\partial}{\partial t} D\tag{M4: Ampère} \\
          B = \mu_0 \mu_r H, \ \ \ \ D = \epsilon_0 \epsilon_r E \notag
        \end{align}
      \end{block}
      %\[ \text{div} j = - \frac{\partial \rho}{\partial t} \tag{Kontinuitätsgleichung} \]
      \neuesymbole
    }

  \subsection{Licht als elektromagnetische Welle}
    \begin{frame}
      \frametitle{Herleitung der Wellengleichung}
      \framesubtitle{Was charakterisiert eine Welle?}
      \only<1>{
        \begin{itemize}
          \item Eine Welle ist im Raum veränderlich
          \item Eine Welle bewegt sich mit der Zeit $t$ und einer Geschwindigkeit $v$ im Raum fort.
        \end{itemize}
                
        Wenn $A(x,t)$ die Amplitude am Ort $x$ zur Zeit $t$ darstellt, muss sie deshalb von folgender Form sein:
        \[ A(x,t) = f(x-vt) \]
        Hierbei ist $f$ eine beliebige zweimal differenzierbare Funktion.
      }
      \only<2>{
        \[ A(x,t) = f(x-vt) \]
        Um zur die Wellengleichung zu gelangen, differenzieren wir die Funktion $A$ jeweils zweimal nach Ort und Zeit.
        \[ \frac{d}{dt}A = \frac{d}{d(x-vt)} f(x-vt) \cdot (-v), \ \ \ \frac{d^2}{dt^2}A = \frac{d^2}{d(x-vt)^2} f(x-vt) \cdot (-v)^2 \]
        \[ \frac{d^2}{dx^2}A = \frac{d^2}{d(x-vt)^2} f(x-vt) \cdot 1^2 \]
      }
      \only<3>{
        \[ 
          \frac{d^2}{dt^2}A = \frac{d^2}{d(x-vt)^2} f(x-vt) \cdot (-v)^2, \ \ \ 
          \frac{d^2}{dx^2}A = \frac{d^2}{d(x-vt)^2} f(x-vt) \cdot 1^2
        \]
        Wenn wir hier den Term $\frac{d^2}{d(x-vt)^2} f(x-vt)$ eliminieren, erhalten wir die Wellengleichung.
        \begin{block}{Wellengleichung}
          \[
            \frac{1}{v^2}\frac{d^2}{dt^2} A = \frac{d^2}{dx^2} A
            \tag{Wellengleichung}
          \]
        \end{block}
      }
    \end{frame}
    
    \begin{frame}
      \frametitle{Rolle für den Elektromagnetismus}
      \framesubtitle{Aus den Maxwellgleichungen folgt eine Wellengleichung.}
            
      \only<1>{
        \neuesymbole
        Keine Anregung: $\rho = 0$, $j=0$ \\
        Materialgleichungen: $D = \epsilon E, \ \ \ B = \mu H$ \\
        Maxwellgleichungen: $\nabla \ D = \rho, \ \ 
          \nabla \ B = 0, \ \
          \nabla \times E = - \frac{\partial}{\partial t} B, \ \
          \nabla \times H = j + \frac{\partial}{\partial t} D $
        
        \vspace{0.3cm}
        Damit erhalten wir aus den Maxwellgleichungen:
      }
      \only<1-2>{
        \[
          \nabla \times E = -\frac{\partial}{\partial t} B, \ \ \ \ \
          \nabla \times \underbrace{\frac{1}{\mu} B}_H = \underbrace{\epsilon \frac{\partial}{\partial t} E}_D, \ \ \ \ \ 
          \nabla E = 0 
        \]
      }
      \only<2>{
        Wir differenzieren die mittlere Gleichung
        \[
          \nabla \times \frac{\partial}{\partial t} B = \epsilon \mu \frac{\partial^2}{\partial t^2} E
        \]
        und setzen die linke ein:
        \[
          - \underbrace{\nabla \times \nabla \times E}_
            {\nabla (\nabla E) - (\nabla\nabla) E } 
          = \epsilon\mu \frac{\partial^2}{\partial t^2} E
        \]
        Hier kommt die rechte Gleichung ins Spiel: $\nabla E = 0$\\
        Deshalb wird aus obiger Gleichung:
      }
      \only<2-3>{
        \[
          \nabla^2 E = \epsilon\mu \frac{\partial^2}{\partial t^2} E
        \]
      }
      \only<3>{
        \[
          \frac{\partial^2}{\partial x^2} A = \frac{1}{v^2}\frac{\partial^2}{\partial t^2} A
          \tag{Vergleich: Wellengleichung}
        \]        
        \begin{block}{Koeffizientenvergleich}
          \[
            \epsilon\mu = \frac{1}{v^2}, \ \ \ \Rightarrow \ \ \ 
            v = \frac{1}{\sqrt{\epsilon\mu}}
          \]
        \end{block}
      }
    \end{frame}
    \begin{frame}
      \frametitle{Rolle des Lichts}
      \framesubtitle{Licht ist genausoschnell.}
      \begin{itemize}[<+->]
        \item Experimentell gemessen: $ v = \frac{1}{\sqrt{\epsilon\mu}} = 310\,740\,000 \frac{\text{m}}{\text{s}} \approx c $
        \item Vermutung: Licht ist eine elektromagnetische Welle.
        \item \zitat{The only use made of light in the experiment was to see the instruments.
            The agreement of the results seems to show that light and magnetism are affections of the same substance, and that light is an electromagnetic disturbance propagated through the field according to electromagnetic laws.
          }{\name{Maxwell}}
          \quelle{Maxwell: \adtef, § 97}
        \item Experimentelle Bestätigung: \name{Hertz} (1887): \\ Erzeugung elektromagnetischer Wellen, \\ die sich optisch so verhalten wie Licht.
      \end{itemize}
    \end{frame}    
    \begin{frame}
      \frametitle{Rolle des Lichts}
      \framesubtitle{Licht ist eine elektromagnetische Welle.}
      \zitat{1864 waren Elektromagnetismus und Licht zu einer Theorie vereinigt.}{\name{Friedrich Hund}}    
      \quelle{Hund, S.\,57}
    \end{frame}    
    
\section{Historische Bemerkungen}

  \frame{
    \frametitle{Historische Bemerkungen}
    \tableofcontents[sections=2-6,currentsection,hideallsubsections]
  }  
  
  \subsection{Leistungen Maxwells}
    \begin{frame}
      \frametitle{Leistungen Maxwells}
      \quelle{Hermann, S.\,83, S.\,227; Frisius, S.\,201; Gamow, S.\,181}
      \begin{itemize}[<+->]
        \item Erweiterung des Ampèreschen Gesetzes um den Verschiebungsstrom $\Rightarrow$ EM-Wellen möglich
        \item Sichtung Faradays Arbeit
        \item Nahwirkung statt Fernwirkung:\\
          \zitat{Faraday sah mit seinem geistigen Auge Kraftlinien den ganzen Raum durchdringen, wo die Mathematiker Anziehungszentren von Fernkräften sahen.}{\name{Maxwell} im Treatise von 1873}
        \item Verallgemeinerung Faradays Ideen
        \item Loslösung von der mechanischen Anschauung
        \item Höhepunkt der klassischen Physik
      \end{itemize}
    \end{frame}    
  
  \subsection{Auswirkungen und Fortgang}
    \begin{frame}
      \frametitle{Auswirkungen und Fortgang}
      \begin{itemize}
        \only<1-2,4->{
          \item<1-> Vom Äther zu virtuellen Teilchen
          \item<2-> Abstraktere Auffassung von Naturerscheinungen
        }
        \only<3>{
          \item<3> \zitat{Vor Maxwell dachte man sich das Physikalisch-Reale --- soweit es die Vorgänge in der Natur darstellen sollte --- als materielle Punkte, deren Veränderungen nur in Bewegungen bestehen, die durch gewöhnliche Differentialgleichungen beherrscht sind. Nach Maxwell dachte man sich das Physikalisch-Reale durch nicht mechanisch deutbare, kontinuierliche Felder dargestellt, die durch partielle Differentialgleichungen beherrscht werden. Diese Veränderung der Auffassung des Realen ist die tiefgehendste und furchtbarste, welche die Physik seit Newton erfahren hat.}{Albert Einstein \hspace{15cm}}
            \quelle{Simonyi, S.\,349}
        }
        \only<1-2,4->{
          \item<5-> Von Newton zur Relativitätstheorie
          \item<6-> Quantenelektrodynamik (QED)
        }
      \end{itemize}
    \end{frame}    

\section{Schluss}
  \subsection{Résumé zur Lektüre der Originalarbeit}
    \begin{frame}
      \frametitle{Résumé zur Lektüre der Originalarbeit}
      \begin{itemize}[<+->]
        \item Für uns ungewöhnliche Größenbezeichnungen
        \item Hineinversetzen in den Kenntnisstand der Zeit
        \item Wichtig: Überblicksliteratur
        \item Einblick in die Entstehung bedeutender Theorien
        \item Besseres \textbf{Verständnis} der Thematik selbst
      \end{itemize}
    \end{frame}    
  
  \subsection{Verstehen von Zusammenhängen}
    \begin{frame}
      \frametitle{Verstehen von Zusammenhängen}
      \zitat{Für Studierende eines jeglichen Wissenschaftsgebietes ist es von großem Nutzen, die diesbezüglichen Originalwerke zu lesen; die Wissenschaft kann nämlich dann vollständig angeeignet werden, wenn sie sich im Zustand des Entstehens befindet.\\
        Wenn irgendetwas davon, was ich hier niedergeschrieben habe, Studenten dazu verhilft, Faradays Ausdrucksweise und Gedankengänge verstehen zu lernen, werde ich eine meiner Hauptzielsetzungen erreicht haben.}{\name{Maxwell} im Vorwort zum Treatise 1873}
      \quelle{Simonyi, S.\,342}
    \end{frame}        
  
  \subsection{Quellen}
    \begin{frame}[shrink=40]
      \frametitle{Quellen}
      \begin{enumerate}
        \item Maxwell, James Clerk: A Dynamical Theory of the Electromagnetic Field, Philosophical Transactions of the Royal Society of London 155, 459-512 (1865).\\
        \url{http://en.wikipedia.org/wiki/File:A_Dynamical_Theory_of_the_Electromagnetic_Field.pdf}, Stand: Januar 2009
        \newline
        
        \item Nolting, Wolfgang: Grundkurs Theoretische Physik 3 Elektrodynamik, Verlag Springer, 2007
        \item Reinhard, Paul-Gerhard: Vorlesung über Elektrodynamik, 2008
        \newline
        
        \item \textbf{Hund}, Friedrich: Geschichte der physikalischen Begriffe, Spektrum Verlag, 1995
        \item \textbf{Hermann}, Armin: Lexikon Geschichte der Physik A-Z, Aulis Verlag, 1972
        \item Hermann, Armin: \textbf{Große Physiker}, Battenberg, 1960 
        \item \textbf{Gamow}, George: Biographie der Physik, Econ-Verlag
        \item \textbf{Frisius}, Joachim: Von Coulomb bis Einstein, Verlag Harri Deutsch, 2001
        \item \textbf{Simonyi}, Károly: Kulturgeschichte der Physik, Verlag Harri Deutsch
        \item \textbf{Scholte}, Karl-Heinz: Chronologie der Naturwissenschaften, Verlag Harri Deutsch
        \item \textbf{Meyenn}, Karl von: Die Großen Physiker, zweiter Band, Verlag Beck, München
        \newline
        
        \item \url{http://de.wikipedia.org/wiki/James_Clerk_Maxwell}, Stand: 5.\,Feb.\,2009 14:18
        \item \url{http://en.wikipedia.org/wiki/Timeline_of_electromagnetism_and_classical_optics}, Stand: 5.\,Feb.\,2009 23:45
        \item \url{http://en.wikipedia.org/wiki/A_Dynamical_Theory_of_the_Electromagnetic_Field}, Stand: 4.\,Feb.\,2009
      \end{enumerate}    
    \end{frame}

\end{document}

