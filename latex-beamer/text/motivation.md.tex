%!TEX TS-program = ../make.zsh

\begin{frame}[fragile]{Motivation and Scope}
  \begin{column}{0.6\textwidth}
    \image{dom_geometry_features2}

    \vspace{1cm}
    Thesis: \url{https://arxiv.org/abs/1904.08422}
  \end{column}
  \begin{column}{0.4\textwidth}
    \begin{itemize}
      \item Topic: \her{Light-propagation simulation} in vicinity of detector modules, considering:
      \begin{itemize}
        \item \her{ice properties} in vicinity, esp. in hole ice
        \item opaque \her{cables}
        \item non-spherical \her{detector modules} of variable position
      \end{itemize}
      \item Usually: \her{Effective} modification of module \her{sensitivity}
      \item Here: Direct \her{ray-tracing} algorithm in clsim
    \end{itemize}

    \vspace{2cm}
    \vtiny{D-Egg-detector-module image: \href{https://indico.scc.kit.edu/event/215/contributions/890/attachments/641/954/IceCube-Gen2_Peiffer_HAP.pdf}{Pfeiffer, New optical sensors for IceCube-Gen2, 2016}}

  \end{column}
\end{frame}

\begin{frame}[fragile]{Motivation and Scope}
  \begin{column}{0.5\textwidth}
    \vspace*{10mm} \image{angular-acceptance-calibration-measurements}
    
    \vspace*{5mm} Different hole-ice properties lead to angular-acceptance profiles possibly different from the a-priory approximation.

  \end{column}
  \begin{column}{0.5\textwidth}
    \image{asymmetry-example-angular-acceptance-with-comment}
    
    Geometric asymmetries with respect to hole ice impact effective acceptance.

  \end{column}
\end{frame}

\begin{frame}[fragile]{Motivation and Scope}
  Some history:
  \begin{itemize}
    \item Thesis 2018
    \item Kernel compatible, but interface not compatible to current icetray \ $\rightarrow$ \ hard to use
    \item 2021: Attempted port to ppc
    \item today: Compatible to current icetray
  \end{itemize}
\end{frame}

% - this talk about simulation of light propagation in the vincinity of the detector modules
% - certain features make simulation more complicated
%   - ice properties in vincinity of modules
%   - cables beside modules
%   - shape of the different types of detector modules
% - currently: using effective quantities to account for these features, but not directly in ray-tracing algorithms
% - for calibration studies, direct simulation more desirable in order to study the effects of these features rather than making assumptions about them
% - in this talk: presenting improved ray-tracing algorithm that accounts for these features, and highlights the major things to watch out for.

% \begin{frame}[fragile]{Motivation and Scope}
%   \begin{column}{0.6\textwidth}
%     \image{hole-ice-camera}
%   \end{column}
%   \begin{column}{0.4\textwidth}
%     \begin{itemize}
%       \item \her{Hole ice} is the refrozen water in the drill holes around the detector modules
%       \item possibly \her{different optical properties} than surrounding bulk ice
%       %\item special kinds: \\ drill-hole ice \\ bubble column
%       \item Implement \her{hole ice in photon-propagation simulation} in order to improve detector calibration
%     \end{itemize}
%
%     \vspace{2cm}
%     \vtiny{Images (a) to (g) show a time series of the freeze-in process. Image (h) shows has been taken several years after the freeze-in process.\\ \par Image sources: Resconi, Rongen, Krings: The Precision Optical CAlibration Module for IceCube-Gen2: First Prototype, 2017. Finley et al.: Freezing in the IceCube camera in string 80, 22 Dec - 1st Jan. 2011. Rongen: The 2018 Sweden Camera run — light at the end of the ice,  2018.}
%   \end{column}
% \end{frame}

% \begin{frame}[fragile]{Motivation and Scope}
%
%   \begin{columns}
%     \begin{column}{0.5\textwidth}
%       %\image{cross-check-71-steamshovel}
%       \image{flasher-steamshovel-single-received-photon}
%
%       \footnotesize Simulation: Incoming photon (red) hits a large hole-ice cylinder
%       with small scattering length
%     \end{column}
%     \begin{column}{0.5\textwidth}
%
%       \begin{itemize}
%
%         \item No direct hole-ice simulation in standard simulation, \\ only angular sensitivity approximation
%           \note[item]{
%             clsim approximates hole ice using a convolution function for the angular acception.
%             \item e.g. photons hitting a dom from below are made more unlikely to be detected.
%             \item but no actual simulation of the changed ice properties.
%           }
%           \begin{itemize}
%             \follows No asymmetries possible, e.g. DOM position relative to hole ice
%             \note[item]{i.e. we can't have asymmetries like shifted DOM positions relative to the hole ice.}
%           \end{itemize}
%
%         \vspace*{1cm}
%         \item implement direct hole-ice simulation in clsim \\ in order to improve detector calibration
%           \note[item]{that's why I'm trying to implement propagation through cylinders with changed ice properties.}
%
%       \end{itemize}
%
%     \end{column}
%   \end{columns}
% \end{frame}

