%!TEX TS-program = ../make.zsh

\begin{frame}[fragile]{Motivation and Scope}
  \begin{column}{0.6\textwidth}
    \image{hole-ice-camera}
  \end{column}
  \begin{column}{0.4\textwidth}
    \begin{itemize}
      \item \her{Hole ice} is the refrozen water in the drill holes around the detector modules
      \item possibly \her{different optical properties} than surrounding bulk ice
      %\item special kinds: \\ drill-hole ice \\ bubble column
      \item Implement \her{hole ice in photon-propagation simulation} in order to improve detector calibration
    \end{itemize}

    \vspace{2cm}
    \vtiny{Images (a) to (g) show a time series of the freeze-in process. Image (h) shows has been taken several years after the freeze-in process.\\ \par Image sources: Resconi, Rongen, Krings: The Precision Optical CAlibration Module for IceCube-Gen2: First Prototype, 2017. Finley et al.: Freezing in the IceCube camera in string 80, 22 Dec - 1st Jan. 2011. Rongen: The 2018 Sweden Camera run — light at the end of the ice,  2018.}
  \end{column}
\end{frame}

% \begin{frame}[fragile]{Motivation and Scope}
%
%   \begin{columns}
%     \begin{column}{0.5\textwidth}
%       %\image{cross-check-71-steamshovel}
%       \image{flasher-steamshovel-single-received-photon}
%
%       \footnotesize Simulation: Incoming photon (red) hits a large hole-ice cylinder
%       with small scattering length
%     \end{column}
%     \begin{column}{0.5\textwidth}
%
%       \begin{itemize}
%
%         \item No direct hole-ice simulation in standard simulation, \\ only angular sensitivity approximation
%           \note[item]{
%             clsim approximates hole ice using a convolution function for the angular acception.
%             \item e.g. photons hitting a dom from below are made more unlikely to be detected.
%             \item but no actual simulation of the changed ice properties.
%           }
%           \begin{itemize}
%             \follows No asymmetries possible, e.g. DOM position relative to hole ice
%             \note[item]{i.e. we can't have asymmetries like shifted DOM positions relative to the hole ice.}
%           \end{itemize}
%
%         \vspace*{1cm}
%         \item implement direct hole-ice simulation in clsim \\ in order to improve detector calibration
%           \note[item]{that's why I'm trying to implement propagation through cylinders with changed ice properties.}
%
%       \end{itemize}
%
%     \end{column}
%   \end{columns}
% \end{frame}

