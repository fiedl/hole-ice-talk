%!TEX TS-program = ../make.zsh

\begin{frame}[fragile]{How does it work?}

  \begin{columns}
    \begin{column}{0.5\textwidth}
      \begin{overlayarea}{\textwidth}{\textheight}
        \vspace*{2cm}
        \only<1>{\image{how-does-it-work-001}}
        \only<2>{\image{how-does-it-work-002}}
        \only<3>{\image{how-does-it-work-003}}
        \only<4>{\image{how-does-it-work-004}}
      \end{overlayarea}
    \end{column}
    \begin{column}{0.5\textwidth}

      \begin{itemize}
        \item Photon scattering points $A$ and $B$
        \item<1> \her{Naive algorithm}: Propagate photon small distance $\delta x$ in each simulation step and randomize whether the photon will scatter in this step (easy to implement local properties)
        \item<2> \her{Standard clsim} algorithm: Randomize geometric distance to next scattering point and propagate from $A$ to $B$ in one simulation step
        \item<3> \her{Ice layers} in clsim: Randomize number of scattering lengths between $A$ and $B$ as budget and calculate geometric distance by spending the budget over the ice layers
        \item<4> \her{New: Hole ice} in clsim: Generalize budget algorithm to support cylinders (with distinct scattering and absorption lengths) in addition to ice layers.
      \end{itemize}

    \end{column}
  \end{columns}

\end{frame}