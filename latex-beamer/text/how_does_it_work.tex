%!TEX TS-program = ../make.zsh

\begin{frame}[fragile]{How does it work?}

  \begin{columns}
    \begin{column}{0.5\textwidth}
      \only<1>{\image{algorithm-trajectory}}
      \only<2>{\image{distance-correction}}
    \end{column}
    \begin{column}{0.5\textwidth}

      \begin{itemize}
        \item<1> In photon propagation simulation, one simulation step consists of everything between two scatterings, i.a.
          \begin{itemize}
            \item randomizing the distance to the next scattering point
            \item randomizing the scattering angle
            \item moving the photon to the next scattering point
            \item checking for absorption
            \item checking for detection at a DOM
          \end{itemize}

        \item<2> Hole ice simulation adds another task to each simulation step:
          \begin{itemize}
            \item Calculate the portion of the photon trajectory in the step that runs through hole ice
            \item Correct the distance to the next scattering point for the changed ice properties within the hole ice
            \item Correct the distance to absorption as well
          \end{itemize}
      \end{itemize}

    \end{column}
  \end{columns}
  \source{\url{https://github.com/fiedl/hole-ice-study#how-does-it-work}}

\end{frame}