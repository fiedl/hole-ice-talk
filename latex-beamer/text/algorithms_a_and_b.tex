%!TEX TS-program = ../make.zsh

\begin{frame}[fragile]{Approach A (naive)}

  \begin{columns}
    \begin{column}{0.5\textwidth}
      \image{distance-correction}
    \end{column}
    \begin{column}{0.5\textwidth}
      \begin{itemize}
        \item \her{Minimal extension} to existing kernel
        \item Hole ice as \her{correction for each scattering step}
        \item Pros:
        \begin{itemize}
          \item Small surface area of implementation
          \item Standard \alert{clsim (well-tested) almost untouched}
          \item Very testable using unit tests
        \end{itemize}
        \item Cons:
        \begin{itemize}
          \item Hole-ice \alert{properties relative} to bulk ice
          \item Does \alert{not work with nested} cylinders
        \end{itemize}
      \end{itemize}
    \end{column}
  \end{columns}

  \source{https://github.com/fiedl/hole-ice-study/issues/45}

\end{frame}

\begin{frame}[fragile]{Approach B (the one to go)}

  \begin{columns}
    \begin{column}{0.5\textwidth}
      \image{how-does-it-work-004}
    \end{column}
    \begin{column}{0.5\textwidth}
      \begin{itemize}
        \item Treat hole ice and cables as \alert{media with ice optical properties}
        \item Generalize ice-layer algorithm
        \item Pros:
        \begin{itemize}
          \item Supports \alert{nested cylinders} and \alert{cables}
        \end{itemize}
        \item Cons:
        \begin{itemize}
          \item Needed to \alert{rewrite} some of the existing propagation kernel
          \item i.e. needed lots of \alert{statistical cross checks} to make sure everything works
        \end{itemize}
      \end{itemize}
    \end{column}
  \end{columns}

  \source{https://github.com/fiedl/hole-ice-study/issues/45}

\end{frame}

