%!TEX TS-program = ../make.zsh

\begin{frame}[fragile]{How approach B work?}

  \begin{columns}
    \begin{column}{0.4\textwidth}
      \image{how-does-it-work-004}
    \end{column}
    \begin{column}{0.6\textwidth}

      \metroset{block=fill}
      \begin{block}{Within standard clsim kernel}
        \begin{itemize}
          \setlength\itemsep{-0.5em}
          \item Take current photon position
          \item and properties of ice layers
          \item Loop over layers
          \item Calculate physical distance to next interaction
        \end{itemize}
      \end{block}

      \vspace*{-1em}
      \begin{center} $\Downarrow$ \end{center}
      \vspace*{-1em}

      \begin{block}{Replace with: \texttt{apply\_propagation\_through\_media}}
        \begin{itemize}
          \setlength\itemsep{-0.5em}
          \item Take current photon position
          \item Define arrays:\\
            \texttt{distances\_to\_medium\_changes}\\
            \texttt{local\_scattering\_lengths}\\
            \texttt{local\_absorption\_lengths}
          \item Add ice layers to these arrays
          \item Add hole-ice and cable cylinders to these arrays
          \item Sort arrays by ascending distance from photon
          \item Loop over arrays
          \item and calculate physical distance to next interaction
        \end{itemize}
      \end{block}


    \end{column}
  \end{columns}

  \source{\newline https://github.com/fiedl/clsim/tree/sf/hole-ice-2018/resources/kernels/lib}

\end{frame}
