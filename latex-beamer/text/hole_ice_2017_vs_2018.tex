%!TEX TS-program = ../make.zsh

\begin{frame}[fragile]{Two hole-ice algorithms}

  \begin{tabelle}{l|L|L}
    & \textbf{Algorithm (a)} & \textbf{Algorithm (b)} \\
    \hline
    Approach
      & Leave clsim medium propagation as it is and add \textbf{hole-ice effects as correction} afterwards
      & Unify clsim medium propagation through layers and hole ice: Treat them as \textbf{generic medium changes} \\
    Hole-ice properties
      & defined relative to bulk-ice properties
      & defined absolute \\
    Pros
      &
        \begin{itemize}
          \item[+] Small surface area of hole-ice code, i.e. well testable through unit tests
          \item[+] Standard clsim almost untouched
        \end{itemize}
      &
        \begin{itemize}
          \item[+] Supports nested cylinders and cables
        \end{itemize}
      \\
    Cons
      &
        \begin{itemize}
          \item[--] Current understanding of hole-ice suggests defining hole-ice properties absolute rather than relative
        \end{itemize}
      &
        \begin{itemize}
          \item[--] Needed rewrite of clsim's medium-propagation code
          \item[--] Ice tilt and ice anisotropy not re-implemented, yet \tiny (\href{https://github.com/fiedl/hole-ice-study/issues/48}{Issue \#48}) \small
        \end{itemize}
  \end{tabelle}

  \source{https://github.com/fiedl/hole-ice-study/issues/45}

\end{frame}