%!TEX TS-program = ../make.zsh

\begin{frame}[fragile]{Trying out different hole-ice scattering lengths}

  The exact optical properties of the hole ice are unknown. With the simulation, one can try out different properties, e.g. scattering length.
  \vspace{1cm}
  \begin{columns}
    \begin{column}{0.5\textwidth}
      \image{example-sca0-1-steamshovel-color}

      \small
      Scattering length $\lambda_\text{sca,hole-ice} = 10^{-1}\, \lambda_\text{sca,bulk}$.
      Absorption length $\lambda_\text{abs,hole-ice} = \lambda_\text{sca,bulk}$.

    \end{column}
    \begin{column}{0.5\textwidth}
      \image{example-sca0-001-steamshovel-color}

      \small
      Scattering length $\lambda_\text{sca,hole-ice} = 10^{-3}\, \lambda_\text{sca,bulk}$.
      Absorption length $\lambda_\text{abs,hole-ice} = \lambda_\text{sca,bulk}$.

    \end{column}
  \end{columns}

  \vspace{0.7cm}
  \small
  Animation on youtube: \url{https://youtu.be/BhJ6F3B-I1s}

  \tiny{View from top onto a detector module within a hole-ice cylinder. Colors indicate simulation steps, i.e. number of scatterings relative to the total number until absorption. \color{red}{Red}: Photon just created, \color{blue}{blue}: Photon about to be absorbed.}

  \source{\url{https://github.com/fiedl/hole-ice-study/issues/39}}

\end{frame}

\note[itemize]{
  \item Photon wave length: 340\,nm (UV)
  \item DOM position: $(-256.02301025390625, -521.281982421875, 500.0)$
  \item Scattering length in bulk ice at this DOM: $1.78\,\text{m} \pm 1.79\,\text{m}$
  \item Absorption length in bulk ice at this DOM: about 150\,m
  \item The last example is rather extreme. No sharp border in current understanding of hole ice. But I don't know the correct ice properties of the hole ice.
  \follows Need a way to compare these simulations to other studies.
  \follows e.g. Angular acceptance plots
}
