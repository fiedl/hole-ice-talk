%!TEX TS-program = ../make.zsh

\begin{frame}[fragile]{DOM oversizing with hole ice?}

  \begin{columns}
    \begin{column}{0.5\textwidth}
      \begin{overlayarea}{\textwidth}{\textheight}
        \vspace*{2cm}
        \only<1>{\image{dom-oversizing-001}}
        \only<2>{\image{dom-oversizing-002}}
        \only<3>{\image{dom-oversizing-003}}
        \only<4>{\image{dom-oversizing-004}}
        \only<5>{\image{dom-oversizing-005}}
      \end{overlayarea}
    \end{column}
    \begin{column}{0.5\textwidth}
      \begin{itemize}
        \item Consider a DOM displaced relative to the bubble column, such that
        \begin{itemize}
          \item a photon from one direction would hit the DOM,
          \item<2-> a photon from another direction might be deflected by the bubble column.
        \end{itemize}
        \item<3-> Consider a sphere with arbitrary radius, e.g. 5\m or 10\m.
        \item<4-> In detailled simulations with direct propagation, record impact position, impact direction, hole-ice displacement, hole-ice azimuthal position, and count hits.
        \item<5> In simulations without direct propagation, just intersect the outer sphere and use the hit probability from the table.
      \end{itemize}
    \end{column}
  \end{columns}

  \source{\url{https://github.com/fiedl/hole-ice-study/issues/116}}

\end{frame}