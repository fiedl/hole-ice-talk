%!TEX TS-program = ../make.zsh

\begin{frame}[fragile]{Angular acceptance}

  For each angle $\eta$, shoot photons onto the DOM and count hits.

  \begin{columns}
    \begin{column}{0.5\textwidth}
      \image{angular-acceptance}
    \end{column}
    \begin{column}{0.5\textwidth}

      \only<1>{
        \begin{figure}
          \image{ice-paper-fig7}
          \caption*{Angular acceptance \textit{reference curves}. The nominal model is based on lab measurement, the hole ice curve on previous simulations.}
        \end{figure}
      }

      \begin{onlyenv}<2>
        \image{angular-acceptance-example-plane-waves}

        \begin{smallbash}
          $ICESIM/env-shell.sh; cd $HOLE_ICE_STUDY/scripts/AngularAcceptance
          ./run.rb --scattering-factor=0.1 --absorption-factor=1.0 --distance=1.0 --plane-wave --number-of-photons=1e5 --angles=0,10,20,30,32,45,60,75,90,105,120,135,148,160,170,180] --number-of-runs=2 --number-of-parallel-runs=2
          open results/current/plot_with_reference.png
        \end{smallbash}
      \end{onlyenv}
    \end{column}
  \end{columns}

  \source{Image: Martin Rongen, Calibration Call 2015-11-06, DARD Update, Slide 9 \\ Plot: \href{https://arxiv.org/abs/1301.5361}{Measurement of South Pole ice transparency with the IceCube LED calibration system}, 2013, figure 7. See also: \url{https://github.com/fiedl/hole-ice-study/issues/10}}

\end{frame}

\note[itemize]{
  \item One way to compare the new simulation to existing results, is to plot angular-acceptance curves.
  \item I.e. for each angle $\eta$, which is the angle between the starting direction of the photon and the column axis, shoot photons onto the DOM, propagate them in simulation and count hits.
  \item The current hole-ice approximations are convolutions onto the DOM angular acceptance.
  \item This is an example using the new hole-ice simulation with arbitrary ice parameters (data points) compared to the old reference curve (red).
}