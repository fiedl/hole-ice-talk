%!TEX TS-program = ../make.zsh

\begin{frame}[fragile]{Performance}

  Time measurement: Propagating $10^5$ photons on CPU

  \begin{columns}
    \begin{column}{0.6\textwidth}
      \image{performance-medium-changes}

      \begin{smallbash}
        $ICESIM/env-shell.sh
        cd $HOLE_ICE_STUDY/scripts/AngularAcceptance
        time ./run.rb --distance=1.0 --number-of-runs=1 --number-of-parallel-runs=1 --cpu --angle=45 --plane-wave --number-of-photons=1e5
      \end{smallbash}

    \end{column}
    \begin{column}{0.4\textwidth}
      \begin{itemize}
        \item Medium propagation features (hole ice, layers) have no measurable performance impact.
        \item Hole-ice-2017 and hole-ice-2018 algorithms have no measurable performance impact for scattering lengths comparable to bulk-ice scattering ($\lambda_s = 20\m$).
        \item Performance drop can be seen when lowering the scattering length, i.e. increasing the number of simulation steps ($\lambda_s = 3\mm$).
      \end{itemize}
    \end{column}
  \end{columns}

\source{https://github.com/fiedl/hole-ice-study/issues/49}

\end{frame}